\chapter{Introduction}
\label{chap:introduction}

\section{Overview}

The field of integrated circuit (IC) design has traditionally relied on proprietary tools and resources, often posing a barrier to entry for students, researchers, and hobbyists due to their high costs. This project, titled "Microelectronic Circuit Design for Neural Interfaces in 0.18um CMOS Technology Using an Open Source VLSI Ecosystem," seeks to challenge this paradigm by leveraging open-source tools for the design and implementation of a CMOS inverter-based amplifier. The ultimate goal is to demonstrate that high-quality IC designs can be achieved through accessible and collaborative open-source resources, thereby promoting innovation and democratizing technology.

\section{Significance of Open-Source Tools}

Open-source tools are pivotal in democratizing technology and fostering an inclusive environment for technological advancements. They provide a cost-effective platform for exploring and innovating in IC design, making it accessible to a broader audience. This project underscores the potential of open-source tools to reduce development costs, encourage collaboration, and accelerate the pace of technological advancements in the field of microelectronics.

\section{Project Objectives}

The primary objectives of this project are multifaceted:
\begin{itemize}
    \item \textbf{Design and Implementation:} Develop a CMOS inverter-based amplifier that adheres to specific design specifications.
    \item \textbf{Tool Utilization:} Demonstrate the capabilities of various open-source tools throughout the IC design workflow.
    \item \textbf{Educational Resource:} Create a comprehensive guide and resource for future projects and studies in IC design using open-source tools.
    \item \textbf{Community Contribution:} Contribute to the open-source community by documenting the design process and sharing the project files.
\end{itemize}

\section{Scope of the Project}

This project encompasses the entire IC design workflow, from initial concept to final layout, utilizing open-source tools such as:
\begin{itemize}
    \item \textbf{Xschem:} For schematic capture and design.
    \item \textbf{Ngspice:} For circuit simulation and analysis.
    \item \textbf{Magic:} For layout design and verification.
    \item \textbf{Netgen:} For layout versus schematic (LVS) checks.
    \item \textbf{SkyWater SKY130 PDK:} The process design kit used to ensure the design meets manufacturing standards.
\end{itemize}

\section{Amplifier Design Basis}

The amplifier designed in this project is inspired by the architecture presented in Figure 3.a of the paper "Extracellular Recording of Entire Neural Networks Using a Dual-Mode Microelectrode Array With 19,584 Electrodes and High SNR." This amplifier is intended to serve as a pixel amplifier in an Active Pixel Sensor (APS) readout circuit on a microelectrode array. The main function of this amplifier is to enhance the signal-to-noise ratio (SNR) for extracellular recordings of neural networks. In the context of a neural interface, the amplifier must reliably amplify the small signals generated by neural activity while maintaining a high SNR to ensure accurate and clear signal readings. This makes the design crucial for high-fidelity neural recordings and subsequent data analysis.

\section{Structure of the Report}

The report is organized into the following chapters:
\begin{itemize}
    \item \textbf{Introduction:} Provides an overview and context for the project.
    \item \textbf{Workflow:} Details the step-by-step design process.
    \item \textbf{Documentation of the Design Process:} Explains how the design process was documented in the wiki, including tools presentation, installation steps, and additional resources.
    \item \textbf{CMOS Inverter-Based Amplifier:} Discusses the design and implementation of the amplifier in detail.
    \item \textbf{Conclusion:} Summarizes the project outcomes and lessons learned.
    \item \textbf{Glossary:} Defines key terms and acronyms.
    \item \textbf{Appendices:} Includes supplementary materials and additional resources.
\end{itemize}

\section{Expected Outcomes}

By the conclusion of this project, we aim to achieve the following outcomes:
\begin{itemize}
    \item Successfully design and simulate a CMOS inverter-based amplifier.
    \item Create a fully functional layout ready for fabrication.
    \item Validate the design through extensive simulation and verification steps.
    \item Document the entire design process to serve as a reference for future projects.
\end{itemize}

In summary, this project not only aims to demonstrate the feasibility of high-quality IC design using open-source tools but also seeks to contribute valuable resources to the open-source community and pave the way for future innovations in the field of microelectronics.


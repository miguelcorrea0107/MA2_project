\chapter{Introduction}
\label{chap:introduction}

Traditionally, the field of integrated circuit (IC) design has relied heavily on proprietary tools and resources, which can be prohibitively expensive. This project, titled "Microelectronic Circuit Design for Neural Interfaces in 0.18um CMOS Technology Using an Open Source VLSI Ecosystem," aims to challenge this paradigm by leveraging open-source tools to design and implement a CMOS inverter-based amplifier. The primary goal is to demonstrate that high-quality IC designs can be achieved through accessible and collaborative open-source resources, thereby promoting innovation and democratizing technology.

Open-source tools play a crucial role in democratizing technology and fostering an inclusive environment for technological advancements. They offer a cost-effective platform for exploring and innovating in IC design, making it accessible to a broader audience. This project underscores the potential of open-source tools to reduce development costs, encourage collaboration, and accelerate technological advancements in microelectronics.

The primary objectives of this project are to:
\begin{itemize}
\item Develop a CMOS inverter-based amplifier that adheres to specific design specifications.
\item Demonstrate the capabilities of various open-source tools throughout the IC design workflow.
\item Create a comprehensive guide for future projects and studies in IC design using open-source tools.
\item Enhance the BSSE group's understanding and experience with open-source technology and Process Design Kits (PDKs).
\end{itemize}

This project encompasses the entire IC design workflow, from initial concept to final layout, utilizing open-source tools such as Xschem for schematic capture, Ngspice for circuit simulation, Magic for layout design, Netgen for layout versus schematic (LVS) checks, and the SkyWater SKY130 PDK to ensure the design meets manufacturing standards.

The amplifier designed in this project is inspired by the architecture presented in the paper by \textcite{Yuan_Hierlemann_Frey_2021}.This amplifier is intended to serve as a pixel amplifier in an Active Pixel Sensor (APS) readout circuit on a microelectrode array. Its primary function is to enhance the signal-to-noise ratio (SNR) for extracellular recordings of neural networks, which is crucial for high-fidelity neural recordings and subsequent data analysis.

In addition to demonstrating the feasibility of using open-source tools, this project provides significant internal benefits to the BSSE group \parencite{Department_of_Biosystems_Science_and_Engineering_2024}). It enhances the group's understanding and experience with open-source technology and PDKs, enabling future projects to leverage these resources effectively. While the project may not demonstrate the full potential of open-source technology, it establishes a solid foundation for the BSSE group to build upon.

The microelectronics aspect of this project includes detailed explanations of the requirements, state of the art, and applications of CMOS inverter-based amplifiers. This includes exploring the specific design criteria necessary for neural interface applications, understanding current advancements in amplifier technology, and applying this knowledge to create a robust and functional design.

The report is organized into the following chapters:
\begin{itemize}
\item \textbf{Introduction:} Provides an overview and context for the project.
\item \textbf{Design Workflow:} Details the step-by-step design process using open-source tools.
\item \textbf{CMOS Inverter-Based Amplifier Design:} Discusses the design and implementation of the amplifier in detail.
\item \textbf{Conclusion:} Summarizes the project outcomes and lessons learned.
\item \textbf{Appendices:} Includes supplementary materials and additional resources.
\end{itemize}

By the conclusion of this project, we aim to successfully design and simulate a CMOS inverter-based amplifier, create a fully functional layout ready for fabrication, validate the design through extensive simulation and verification steps, and document the entire design process to serve as a reference for future projects.

In summary, this project not only aims to demonstrate the feasibility of high-quality IC design using open-source tools but also provides significant benefits to the BSSE group by enhancing their experience and understanding of open-source VLSI design.
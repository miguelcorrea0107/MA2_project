\chapter{Conclusion}
\label{chap:conclusion}

This project has successfully demonstrated the feasibility and effectiveness of using open-source tools for the design of integrated circuits (ICs), specifically focusing on a CMOS inverter-based amplifier for neural interface applications. Throughout the project, we meticulously designed, simulated, and verified the amplifier, ensuring it met all specified design goals. The amplifier's design was tailored to meet stringent requirements for power supply voltage, input-referred noise, gain, frequency response, and power consumption.

A significant achievement of this project was the effective integration of several open-source tools, including Xschem, Ngspice, Magic, and Netgen, alongside the SkyWater SKY130 PDK. Extensive simulations and verifications were conducted, such as AC response, noise analysis, transient simulation, and DC characteristics, to confirm the design's reliability and performance. Additionally, the entire design process was thoroughly documented, with all project files made publicly available in a GitHub repository \cite{miguelcorrea0107_2024}, contributing to the broader open-source community.

Throughout this endeavor, we learned several important lessons. Open-source tools have reached a level of maturity that allows them to handle complex IC design tasks, albeit with a need for more setup and troubleshooting compared to proprietary tools. Leveraging the support and resources of the open-source community proved invaluable, and active participation in forums and collaboration with other designers significantly enhanced the design process. Furthermore, the iterative nature of IC design was reinforced, highlighting the importance of frequent simulations and checks, such as DRC and LVS verifications, to catch and correct errors early. Finally, the critical role of thorough documentation was underscored, not only for personal reference but also for knowledge sharing. Detailed records of each design step and decision facilitate future projects and contribute to the collective knowledge base.

Despite the project's successes, several challenges were encountered and overcome. Ensuring compatibility between different tools required meticulous attention, particularly when integrating outputs and inputs across various software. Additionally, while open-source tools are powerful, they sometimes lack the advanced functionalities found in proprietary tools, such as sophisticated simulation models and advanced layout automation. This limitation necessitated creative problem-solving and workarounds to achieve the desired design goals.

This project paves the way for several future endeavors and improvements. Incorporating more advanced design features and exploring other types of amplifiers or ICs can expand the scope and utility of open-source IC design. Moving from design and simulation to the actual fabrication and testing of the amplifier would be a logical next step, providing real-world validation of the design.

In conclusion, this project underscores the potential of open-source tools in the field of IC design, particularly for the BSSE group. By achieving our design goals and documenting the process comprehensively, we aim to contribute valuable knowledge and resources to the field. Open-source design not only democratizes access to advanced technology but also fosters innovation through collaboration and shared learning. The successful outcomes of this project lay a solid foundation for future explorations and advancements in open-source IC design.
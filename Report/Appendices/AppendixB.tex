\chapter{Documentation of the Design Process}
\label{AppendixB}

\section{Overview}

This appendix provides an overview of how the design process for the CMOS inverter-based amplifier was meticulously documented in the project's wiki \cite{ethz_bsse_wiki}. 
The wiki serves as a comprehensive resource, offering detailed information on the workflow, tools, and methodologies employed throughout the project. 
It guides users through every phase of the design, from the initial concept to the final conclusions, ensuring that future designers can replicate and build upon this work.


\section{Wiki Page Structure and Contents}

The main page of the Open-source IC Design (OSICD) wiki is designed to be user-friendly and informative, providing a clear roadmap of the project.
The wiki is organized into several key subpages:

\begin{itemize}
    \item OSICD - Introduction
    \item OSICD - Workflow
    \item OSICD - Tools Presentation and Installation
    \item OSICD - CMOS Inverter-Based Amplifier
    \item OSICD - Conclusion
    \item OSICD - Glossary
    \item OSICD - Appendices
\end{itemize}

The content of the wiki mirrors the structure of this report but is presented in a more didactic and user-friendly way, making it accessible for new users and enhancing the learning experience.

\section{Purpose and Benefits of the Wiki Documentation}

The primary purpose of the wiki documentation is to create a comprehensive and accessible resource for future designer within the BEL group. 
By meticulously documenting each stage of the design process, the wiki ensures that the knowledge gained during this project is preserved and can be leveraged by others. The benefits of this documentation include:

\begin{itemize}
    \item \textbf{Educational Resource:} The wiki serves as a detailed educational guide for students and researchers new to IC design.
    \item \textbf{Knowledge Sharing:} By making the design process transparent, the wiki promotes knowledge sharing and collaboration within the open-source community.
    \item \textbf{Replication and Improvement:} Future projects can replicate the design process and build upon the work done, leading to continuous improvement and innovation.
    \item \textbf{Troubleshooting and Support:} Detailed documentation helps in troubleshooting issues and provides support to users facing similar challenges.
\end{itemize}

In conclusion, the documentation of the design process in the wiki not only supports this project but also contributes significantly to the open-source IC design knowledge within the BEL group.